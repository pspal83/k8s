Kubernetes CKA Exam Question.


1. Craete a new pod called admin-pod with images busybox. Allow the pod o be able to set system_time. The container should sleep for 3200 seconds.
2. A kubeconfig file called test.kubeconfig has been created in /root/TEST. There is something wrong with the configuration. Troubleshoot and fix it.
3. Craete a new deployment called web-project-268, with image nginx:1.16 and 1 replica. Next upgrade the deployment to version 1.17 using rolling update. Make sure that the version upgrade is recorded in the resource annotation.
4. Monitor the logs of pod foo and: Extract log lines corresponding to error unable-to-access-website Write them to /opt/KULM00201/foo
5. List all persistent volumes sorted by capacity, saving the full kubectl output to /opt/KUCC00102/volume_list. Use kubectl 's own functionality for sorting the output, and do not manipulate it any further.
6. Ensure a single instance of pod nginx is running on each node of the Kubernetes cluster where nginx also represents the Image name which has to be used. Do not override any taints currently in place.
Use DaemonSet to complete this task and use ds-kusc00201 as DaemonSet name.
7. Perform the following tasks: Add an init container to hungry-bear (which has been defined in spec file /opt/KUCC00108/pod-spec-KUCC00108.yaml) The init container should create an empty file named
/workdir/calm.txt If /workdir/calm.txt is not detected, the pod should exit Once the spec file has been updated with the init container definition, the pod should be created
8. Create a pod named kucc8 with a single app container for each of the following images running inside (there may be between 1 and 4 images specified): nginx + redis + memcached.
9. Create a new service account with the name pvviewer. Grant this Service account access to list all PersistentVolumes in the cluster by creating an appropriate cluster role called 
pvviewer-role and ClusterRoleBinding called pvviewer-role-binding.
10. 
11. Create snapshot of the etcd running at https://127.0.0.1:2379. Save snapshot into /opt/etcd-snapshot.db.
12. Create a Persistent Volume with the given specification. Volume Name: pv-analytics, Storage: 100Mi, Access modes: ReadWriteMany, Host Path: /pv/data-analytics
13. Taint the worker node to be Unschedulable. Once done, create a pod called dev-redis, image redis:alpine to ensure workloads are not scheduled to this worker node. Finally, 
create a new pod called prod-redis and image redis:alpine with toleration to be scheduled on node01.
14. Set the node named worker node as unavailable and reschedule all the pods running on it. (Drain node)
15. Create a Pod called non-root-pod , image: redis:alpine
16. Create a NetworkPolicy which denies all ingress traffic
17. List all the namespaces in the cluster
18. List all the pods in all namespaces
19. List all the pods in the particular namespace
20. List all the services in the particular namespace
21. List all the pods showing name and namespace with a json path expression
22. Create an nginx pod in a default namespace and verify the pod running
23. Create the same nginx pod with a yaml file
24. Output the yaml file of the pod you just created
25. Output the yaml file of the pod you just created without the cluster-specific information
26. Get the complete details of the pod you just created
27. Delete the pod you just created
28. Delete the pod you just created without any delay (force delete)
29. Create the nginx pod with version 1.17.4 and expose it on port 80
30. Change the Image version to 1.15-alpine for the pod you just created and verify the image version is updated
31. Change the Image version back to 1.17.1 for the pod you just updated and observe the changes
32. Check the Image version without the describe command
33. Create the nginx pod and execute the simple shell on the pod
34. Get the IP Address of the pod you just created
35. Create a busybox pod and run command ls while creating it and check the logs
36. If pod crashed check the previous logs of the pod
37. Create a busybox pod with command sleep 3600
38. Check the connection of the nginx pod from the busybox pod
39. Create a busybox pod and echo message ‘How are you’ and delete it manually
40. Create a busybox pod and echo message ‘How are you’ and have it deleted immediately
41. Create an nginx pod and list the pod with different levels of verbosity
42. List the nginx pod with custom columns POD_NAME and POD_STATUS
43. List all the pods sorted by name
44. List all the pods sorted by created timestamp
45. Create a Pod with three busy box containers with commands “ls; sleep 3600;”, “echo Hello World; sleep 3600;” and “echo this is the third container; sleep 3600” respectively and check the status
46. Check the logs of each container that you just created
47. Check the previous logs of the second container busybox2 if any
48. Run command ls in the third container busybox3 of the above pod
49. Show metrics of the above pod containers and puts them into the file.log and verify
50. Create a Pod with main container busybox and which executes this “while true; do echo ‘Hi I am from Main container’ >> /var/log/index.html; sleep 5; done” and with sidecar container with nginx 
image which exposes on port 80. Use emptyDir Volume and mount this volume on path /var/log for busybox and on path /usr/share/nginx/html for nginx container. Verify both containers are running.
51. Exec into both containers and verify that main.txt exist and query the main.txt from sidecar container with curl localhost
52. Get the pods with label information
53. Create 5 nginx pods in which two of them is labeled env=prod and three of them is labeled env=dev
54. Verify all the pods are created with correct labels
55. Get the pods with label env=dev
56. Get the pods with label env=dev and also output the labels
57. Get the pods with label env=prod
58. Get the pods with label env=prod and also output the labels
59. Get the pods with label env
60. Get the pods with labels env=dev and env=prod
61. Get the pods with labels env=dev and env=prod and output the labels as well
62. Change the label for one of the pod to env=uat and list all the pods to verify
63. Remove the labels for the pods that we created now and verify all the labels are removed
64. Let’s add the label app=nginx for all the pods and verify
65. Get all the nodes with labels (if using minikube you would get only master node)
66. Label the node (minikube if you are using) nodeName=nginxnode
67. Create a Pod that will be deployed on this node with the label nodeName=nginxnode
68. Verify the pod that it is scheduled with the node selector
69. Verify the pod nginx that we just created has this label
70. Annotate the pods with name=webapp
71. Verify the pods that have been annotated correctly
72. Remove the annotations on the pods and verify
73. Remove all the pods that we created so far
74. Create a deployment called webapp with image nginx with 5 replicas
75. Get the deployment you just created with labels
76. Output the yaml file of the deployment you just created
77. Get the pods of this deployment
78. Scale the deployment from 5 replicas to 20 replicas and verify
79. Get the deployment rollout status
80. Get the replicaset that created with this deployment
81. Get the yaml of the replicaset and pods of this deployment
82. Delete the deployment you just created and watch all the pods are also being deleted
83. Create a deployment of webapp with image nginx:1.17.1 with container port 80 and verify the image version
84. Update the deployment with the image version 1.17.4 and verify
85. Check the rollout history and make sure everything is ok after the update
86. Undo the deployment to the previous version 1.17.1 and verify Image has the previous version
87. Update the deployment with the image version 1.16.1 and verify the image and also check the rollout history
88. Update the deployment to the Image 1.17.1 and verify everything is ok
89. Update the deployment with the wrong image version 1.100 and verify something is wrong with the deployment
90. Undo the deployment with the previous version and verify everything is Ok
91. Check the history of the specific revision of that deployment
92. Pause the rollout of the deployment
93. Update the deployment with the image version latest and check the history and verify nothing is going on
94. Resume the rollout of the deployment
95. Check the rollout history and verify it has the new version
96. Apply the autoscaling to this deployment with minimum 10 and maximum 20 replicas and target CPU of 85% and verify hpa is created and replicas are increased to 10 from 1
97. Clean the cluster by deleting deployment and hpa you just created
98. Create a Job with an image node which prints node version and also verifies there is a pod created for this job
99. Get the logs of the job just created
100. Output the yaml file for the Job with the image busybox which echos “Hello I am from job”
101. Copy the above YAML file to hello-job.yaml file and create the job
102. Verify the job and the associated pod is created and check the logs as well
103. Delete the job we just created
104. Create the same job and make it run 10 times one after one
105. Watch the job that runs 10 times one by one and verify 10 pods are created and delete those after it’s completed
106. Create the same job and make it run 10 times parallel
107. Watch the job that runs 10 times parallelly and verify 10 pods are created and delete those after it’s completed
108. Create a Cronjob with busybox image that prints date and hello from kubernetes cluster message for every minute
109. Output the YAML file of the above cronjob
110. Verify that CronJob creating a separate job and pods for every minute to run and verify the logs of the pod
111. Delete the CronJob and verify all the associated jobs and pods are also deleted.
112. List Persistent Volumes in the cluster
113. Create a hostPath PersistentVolume named task-pv-volume with storage 10Gi, access modes ReadWriteOnce, storageClassName manual, and volume at /mnt/data and verify
114. Create a PersistentVolumeClaim of at least 3Gi storage and access mode ReadWriteOnce and verify status is Bound
115. Delete persistent volume and PersistentVolumeClaim we just created
116. Create a Pod with an image Redis and configure a volume that lasts for the lifetime of the Pod
117. Exec into the above pod and create a file named file.txt with the text ‘This is called the file’ in the path /data/redis and open another tab and exec again with the same pod and verifies file exist in the same path.
118. Delete the above pod and create again from the same yaml file and verifies there is no file.txt in the path /data/redis
119. Create PersistentVolume named task-pv-volume with storage 10Gi, access modes ReadWriteOnce, storageClassName manual, and volume at /mnt/data and Create a PersistentVolumeClaim of at 
least 3Gi storage and access mode ReadWriteOnce and verify status is Bound
120. Create an nginx pod with containerPort 80 and with a PersistentVolumeClaim task-pv-claim and has a mouth path "/usr/share/nginx/html"
121. List all the configmaps in the cluster
122. Create a configmap called myconfigmap with literal value appname=myapp
123. Verify the configmap we just created has this data
124. delete the configmap myconfigmap we just created
125. Create a file called config.txt with two values key1=value1 and key2=value2 and verify the file
126. Create a configmap named keyvalcfgmap and read data from the file config.txt and verify that configmap is created correctly
127. Create an nginx pod and load environment values from the above configmap keyvalcfgmap and exec into the pod and verify the environment variables and delete the pod
128. Create an env file file.env with var1=val1 and create a configmap envcfgmap from this env file and verify the configmap
129. Create an nginx pod and load environment values from the above configmap envcfgmap and exec into the pod and verify the environment variables and delete the pod
130. Create a configmap called cfgvolume with values var1=val1, var2=val2 and create an nginx pod with volume nginx-volume which reads data from this configmap cfgvolume and put it on the path /etc/cfg
131. Create a pod called secbusybox with the image busybox which executes command sleep 3600 and makes sure any Containers in the Pod, all processes run with user ID 1000 and with group id 2000 and verify.
132. Create the same pod as above this time set the securityContext for the container as well and verify that the securityContext of container overrides the Pod level securityContext.
133. Create pod with an nginx image and configure the pod with capabilities NET_ADMIN and SYS_TIME verify the capabilities
134. Create a Pod nginx and specify a memory request and a memory limit of 100Mi and 200Mi respectively.
135. Create a Pod nginx and specify a CPU request and a CPU limit of 0.5 and 1 respectively.
136. Create a Pod nginx and specify both CPU, memory requests and limits together and verify.
137. Create a Pod nginx and specify a memory request and a memory limit of 100Gi and 200Gi respectively which is too big for the nodes and verify pod fails to start because of insufficient memory
138. Create a secret mysecret with values user=myuser and password=mypassword
139. List the secrets in all namespaces
140. Output the yaml of the secret created above
141. Create an nginx pod which reads username as the environment variable
142. Create an nginx pod which loads the secret as environment variables
143. List all the service accounts in the default namespace
144. List all the service accounts in all namespaces
145. Create a service account called admin
146. Output the YAML file for the service account we just created
147. Create a busybox pod which executes this command sleep 3600 with the service account admin and verify
148. Create an nginx pod with containerPort 80 and it should only receive traffic only it checks the endpoint / on port 80 and verify and delete the pod.
149. Create an nginx pod with containerPort 80 and it should check the pod running at endpoint / healthz on port 80 and verify and delete the pod.
150. Create an nginx pod with containerPort 80 and it should check the pod running at endpoint /healthz on port 80 and it should only receive traffic only it checks the endpoint / on port 80. verify the pod.
151. Check what all are the options that we can configure with readiness and liveness probes
152. Create the pod nginx with the above liveness and readiness probes so that it should wait for 20 seconds before it checks liveness and readiness probes and it should check every 25 seconds.
153. Create a busybox pod with this command “echo I am from busybox pod; sleep 3600;” and verify the logs.
154. copy the logs of the above pod to the busybox-logs.txt and verify
155. List all the events sorted by timestamp and put them into file.log and verify
156. Create a pod with an image alpine which executes this command ”while true; do echo ‘Hi I am from alpine’; sleep 5; done” and verify and follow the logs of the pod.
157. Create the pod with this kubectl create -f https://gist.githubusercontent.com/bbachi/212168375b39e36e2e2984c097167b00/raw/1fd63509c3ae3a3d3da844640fb4cca744543c1c/not-running.yml. 
The pod is not in the running state. Debug it.
158. This following yaml creates 4 namespaces and 4 pods. One of the pod in one of the namespaces are not in the running state. Debug and fix it. 
https://gist.githubusercontent.com/bbachi/1f001f10337234d46806929d12245397/raw/84b7295fb077f15de979fec5b3f7a13fc69c6d83/problem-pod.yaml.
159. Get the memory and CPU usage of all the pods and find out top 3 pods which have the highest usage and put them into the cpu-usage.txt file
160. Create an nginx pod with a yaml file with label my-nginx and expose the port 80
161. Create the service for this nginx pod with the pod selector app: my-nginx
162. Find out the label of the pod and verify the service has the same label
163. Delete the service and create the service with kubectl expose command and verify the label
164. Delete the service and create the service again with type NodePort
165. Create the temporary busybox pod and hit the service. Verify the service that it should return the nginx page index.html.
166. Create a NetworkPolicy which denies all ingress traffic